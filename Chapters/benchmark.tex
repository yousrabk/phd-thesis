% Chapter 4

\chapter{Benchmarking on Phantom datasets}
\label{chapter:benchmark}
\noindent\makebox[\linewidth]{\rule{0.75\paperwidth}{0.4pt}}
\noindent\makebox[\linewidth]{\rule{0.75\paperwidth}{0.4pt}}

\localtableofcontents % local toc

\noindent\makebox[\linewidth]{\rule{0.75\paperwidth}{0.4pt}}
\noindent\makebox[\linewidth]{\rule{0.75\paperwidth}{0.4pt}}
\newpage

\section{Introduction}
The previous chapters define various ways to solve the inverse problem of MEG/EEG brain imaging techniques. The evaluation of these solvers remain difficult due to the completely unknown ground-truth of the exact localization of the involved sources in each specific task. This limitation is primarily coming from the fact that the recording is done over the scalp and that multiple source configurations can lead to exactly the same measurements over the sensors. So the question is whether all these existing source localization techniques are able to accurately locate the positions and the orientations of current sources in the brain in a real acquisition scenario.

The typical way to answer this question is to perform simulations~\cite{liu2002monte,mosher1993error,liu1998spatiotemporal}. It consists in fixing the number and the location, orientation and amplitude of several dipoles in the brain, generate some simulated data corrupted by some additive noise~\cite{liu2002monte,mosher1993error,liu1998spatiotemporal,de2002estimating,wolters2006influence}. These simulations are unfortunately rarely realistic: they do not take into account the non-ideal nature of the sensors and the errors in the forward model, and they do not take into account the complex noise structure of real measurements. Inaccuracies in the computation of the forward operator are mainly due to approximations in the conductivity values in the head and/or the numerical errors associated with either spherical head approximations or \ac{bem} based on more realistic head geometries.

More sophisticated simulations might be investigated to overcome these issues, yet we propose here to use data collected from an artificial physical object in a real MEG machine. This has the advantage that the results can more closely reflect \textit{in vivo} performance since they include factors that cannot readily be included in simulations, such as environmental noise. 

In order to calibrate each MEG device, artificial objects that mimic the brain activity called "\textit{phantoms}" are constructed by the manufacturers of MEG systems.
% that we use to evaluate our solvers to the MEG/EEG inverse problem~\cite{hazim2015magnetoencephalography}.
They are based on the theoretical description in~\cite{ilmoniemi1985forward} producing realistic data corresponding to complex spatio-temporal current sources including realistic head geometries. In a typical phantom, from 4 to 32 independent current dipoles are present and MEG data are collected separately for each dipole. The true dipole locations and orientations, and the morphology of the brain, skull layers can be extracted from X-ray CT data~\cite{leahy1998study}. One limitation is that such phantoms are not unsuitable for EEG, yet there exists some work on making EEG phantom devices~\cite{hairston2016ballistic}.

This chapter presents a new study to validate localization techniques using different publicly available phantom datasets. Other works have been done by~\cite{hazim2015magnetoencephalography,leahy1998study,baillet2001evaluation} using also real-skull phantoms to investigate the performance of representative methods considering various head models.

The approaches considered in this chapter are mainly those described before in this thesis (see Chapter~\ref{chapter:background}), namely: Dipole fitting (Section~\ref{section_dipfit}-page~\pageref{section_dipfit}), Gamma-Map, RAP-MUSIC, MxNE, irMxNE, TF-MxNE, irTF-MxNE. The methods not defined so far, will be briefly described in the next section.

\section{Phantom datasets}
\subsection{Brainstorm-Elekta dataset}
The description below was taken from the Brainstorm tutorial about the MEG phantom~\cite{tadel2011brainstorm}.\\

A current phantom is provided with Elekta for checking the system performance and can then be used for evaluation of the source localization (Figures~\ref{fig:phantom_head} and~\ref{fig:phantom_in_MEG}). It contains 32 artificial dipoles and four fixed head-position indicator coils. The phantom is based on the mathematical fact that an equilateral triangular line current produces a magnetic field distribution equivalent to that of a tangential current dipole in a spherical conductor, provided that the vertex of the triangle and the origin of the conducting sphere coincide. For a detailed description of how the phantom works, see~\cite{ilmoniemi1985forward}.\\
\\

\begin{figure}[tb]
   \centering
\includegraphics[trim={1cm 8cm 2cm 9cm},width = 0.97\textwidth]{benchmark/phantom_head}
\caption{Reference phantom (RefPhantom) NM24058N (Serial number: 101861) provided by Elekta Oy, Helsinki Finland. 32 built-in simulated dipoles and four presetting head position indicator coils (HPI) (Figure taken from~\cite{hazim2015magnetoencephalography}).}
   \label{fig:phantom_head}
\end{figure}

The phantom dipoles are energized using an internal signal generator which also feeds the HPI coils. An external multiplexer box is used to connect the signal to the individual dipoles. Only one dipole can be activated at a time. The location of the dipole is recorded relatively to the center of the sphere (0,0,0), where X is positive toward the nasion, Y is positive toward the left ear and Z is positive toward the top of the head.\\

The dataset has 32 dipoles, and is sorted into 3 different amplitude levels:
\begin{itemize}
\item Source with \textbf{2000 nAm}. This corresponds to an unrealistically strong 1000 nAm (2000 nAm peak-to-peak) dipole that gives the highest SNR of the experimental source.
\item Source with \textbf{200 nAm}. This is a weaker dipole, closer to the range of amplitudes we can expect in raw data.
\item Source with \textbf{20 nAm}. This represents some of the weakest sources we expect in evoked studies, which require averaging in order to be detected and estimated (\textit{i.e.} generally cannot be seen in single trial analysis).
\end{itemize}

\begin{figure}[tb]
   \centering
\includegraphics[trim={1cm 8cm 2cm 9cm},width = 0.97\textwidth]{benchmark/phantom_in_MEG}
\caption{The phantom is carefully set into the sensor helmet of the probe unit and pushed against the helmet. HPI coil is fitted into outlet under the right gantry side cover (Figure taken from~\cite{hazim2015magnetoencephalography}).}
   \label{fig:phantom_in_MEG}
\end{figure}

\subsection{MNE-Elekta dataset} \label{data:used_data}

The dataset has four dipoles named from 5 to 8, each with a different depth in the phantom head. The dataset contains dipoles with four different amplitudes: 2000\,nAm, 200\,nAm, 100\,nAm and 20\,nAm.

\section{Methodology}

\subsection{Sphere models}
The most commonly used head model assumes that it is made up of a set of nested concentric spheres, each with homogeneous and isotropic conductivity. Under this assumption, both the EEG and MEG problems have well-known closed form solutions~\cite{mosher1999eeg}.

\subsection{Preprocessing}
The two Elekta phantom datasets have had a specific and identical preprocessing. First, we computed the forward operator with a single-shell sphere with origin (0., 0., 0.) and a head radius of 0.08, resulting in a discrete source space with 13782 location and free orientation ($O=3$ orientations). Then we marked some channels as bad for each dataset:
\begin{itemize}
\item For Brainstorm-Elekta dataset, one channel has been marked as bad: "MEG2421".
\item For MNE-Elekta dataset, three channels have been marked as bad: \\"MEG2233", "MEG2422", "MEG0111".
\end{itemize}
Second, Maxwell filtering was used to clean the data to to compensate for external magnetic interferences. The data was then low-pass filtered below 40Hz as we are interested only in frequencies around the simulated burst, which is at 25Hz. Once the filtering is done, we can construct the epochs and the evoked response from the filtered data from -100 ms to 100 ms. The covariance was computed on the baseline from -100 ms to 0. The sampling frequency for both datasets is of 1000Hz.
The MNE-Elekta dataset resulted in 33 epochs before averaging, while the Brainstorm-Elekta dataset resulted in 20 epochs.

\subsection{The selected solvers}
\subsubsection{Dipole fitting}
Dipole fitting assumes that a small number of point-like \ac{ECD}s can describe the measured topography. It optimises the location, the orientation and the amplitude of the model dipoles in order to minimise the difference between the model and the measured topography. A good introduction to dipole fitting is provided by~\cite{scherg1990fundamentals}. A full description is done in Chapter~\ref{chapter:background}-Section~\ref{section_dipfit} (page~\pageref{section_dipfit}).

\subsubsection{$\gamma$-Map}
$\gamma$-map offers a unifying view using a Bayesian perspective on some of the solvers presented here. As it was shown in Chapter~\ref{chapter:bayesian}, the Bayesian formulation of the MEG/EEG source localization consists in choosing priors/hyperpriors, estimation and inference procedure where often it boils down to alternating between the estimation of the source estimate and the estimation of the hyperparameter. The hyperparameter in this method is $\gamma$ which represents covariance components. More details are in~\cite{Wipf-Nagarajan:2009}.

\subsubsection{RAP-MUSIC}
\Ac{RAP-MUSIC} is an extension to \ac{MUSIC} by recursively estimating multiple sources~\cite{mosher1997source,mosher1999source}. In other words, it consists in applying MUSIC successively after removing the contribution of the previously identified sources. In the same way matching pursuit algorithms are used for sparse signal decomposition over dictionary of atoms, the RAP-MUSIC method adopts a greedy strategy to select the relevant dipoles in a dictionary of sources.

\subsubsection{MxNE | irMxNE}
\Ac{MxNE} and \ac{irMxNE} are the convex and the non-convex version of mixed-norms solvers using $\ell_{2,1}$-norm and $\ell_{2,0.5}$-quasinorm as regularizations. For more details, see Chapter~\ref{chapter:background}-Section~\ref{sec:mixed_norms} (page~\pageref{sec:mixed_norms}) and~\cite{gramfort2012mixed,strohmeier-etal:16}.

\subsubsection{TF-MxNE | irTF-MxNE}
\Ac{TF-MxNE} and \ac{irTF-MxNE} are the convex and the non-convex versions of the Time-Frequency mixed-norms using $\ell_{2,1}+\ell_1$-norm and $\ell_{2,0.5}+\ell_{0.5}$-quasinorm as regularizations. For more details, see Chapter~\ref{chapter:background}-Section~\ref{sec:mixed_norms} and~\cite{TF-MxNE,bekhti2016m}.

All solvers except the irTF-MxNE are implemented in the MNE-python package~\cite{mne,mne-python}. The dipole fitting, MUSIC, dSPM, MNE, sLORETA, $\gamma$-map have used the by default version. The regularization parameter of LCMV was set to $reg=1$.
The different hyperparameters of (ir)TF-MxNE were tuned on a grid search. The results below are shown after setting the window size to $wsize=64$, time shift to $ts=4$, $\alpha_{space}=40.$, and $\alpha_{time}=1.$. While for the (ir)MxNE regularization hyperparameter was set to $\alpha_{space}=40.$.

\section{Experimental results}
Three types of errors have been investigated for the different solvers, namely: the position or location error, the orientation error, and the amplitude error. All the solvers investigated here are implemented in the MNE-python package~\cite{mne-python,mne}.

The position error $err_{pos}$ is represented in millimeters (mm), defining the distance between the exact location ($pos_{simulated}\in\RR^3$) of the simulated dipole in the phantom head and the estimated location ($pos_{estimated}\in\RR^3$) as in Equation~\eqref{eq:pos_err}. When the estimated source space contains multiple dipoles, the one having the biggest peak of amplitude is kept and compared to the exact dipole.

\begin{equation}\label{eq:pos_err}
err_{pos} = 10^3 \|pos_{estimated} - pos_{simulated}\|_2
\end{equation}

The orientation error $err_{ori}$ is represented in Radians (Rad), defining the angle between the exact orientation ($ori_{simulated}\in\RR^3$) of the simulated dipole and the estimated one ($ori_{estimated}\in\RR^3$) as in Equation~\eqref{eq:ori_err}. Same for the position error, the best dipole is kept when multiple ones are estimated.

\begin{equation}\label{eq:ori_err}
err_{ori} = \arccos(|\langle ori_{estimated}, ori_{simulated} \rangle|)
\end{equation}

Finally, the amplitude error $err_{amp}$ is represented in penrcentage error (\%), except for some solvers which give source estimates with statistical values and not electrical current values (example: dSPM). This error defines the relative difference between the peak of amplitude ($\max(amp_{estimated})$ with $amp_{estimated}\in\RR^T$) of the estimated dipole and the peak of simulated dipole $amp_{simulated}\in\RR$ (example 1000nAm, 200nAm, ... 20nAm) as in Equation~\eqref{eq:amp_err}.

\begin{equation}\label{eq:amp_err}
err_{amp} = \frac{|\max(amp_{estimated}) - amp_{simulated}|}{amp_{simulated}} * 100
\end{equation}

\subsection{Critical comparison of these MEG/EEG source localization}

Figure~\ref{fig:all_solvers_loc_error} shows position errors obtained with most of the solvers for the four simulated dipoles (5 to 8) (see dataset \ref{data:used_data}) and for the different amplitude levels (20, 100, 200, 2000nAm). It shows less than 1mm error for the unrealistic high \ac{SNR} (peak-to-peak amplitude equal to 1000nAm), but also for 200nAM, and 100nAm. The location error gets worse with the very low SNR (20nAm).

\begin{figure}[h!]
	\centering
    \includegraphics[width=0.95\textwidth]{benchmark/phantom_errors_compare_all_solvers_loc_error}
    \caption{Comparison of the position error between most of the solvers for four different dipoles. \label{fig:all_solvers_loc_error}}
\end{figure}

The dipole fitting approach is very suitable for localizing the neuronal activity when a small number of \ac{ECD}s can describe the data. In this dataset \ref{data:used_data}, each dipole is recorded alone. The errors in orientations are displayed in Figures~\ref{fig:all_solvers_ori_error}.

RAP-MUSIC is an approach based on the MUSIC technique which also performs very well when few ECDs are involved, especially when it is a dataset recording only one dipole at a time. It can be seen as very competitive, \textit{w.r.t.} dipole fitting in Figures~\ref{fig:all_solvers_loc_error}-\ref{fig:all_solvers_ori_error}, showing the errors in position, and orientation respectively. However, for a deep source (dipole 8) combined with a very low SNR (20nAm), the red curve is outside of the box, meaning a location error bigger than 20mm. This is an issue with the signal subspace estimation, where the rank of the data covariance is not well estimated.

The $\gamma$-map which is a Bayesian formulation of the MEG/EEG inverse problem,  performs worse than dipole fitting or RAP-MUSIC for very high and very low SNR. For SNRs in range of realistic data, its location error is upper bounded by 5mm depending on the depth of the studied dipole. The $\gamma$-map is doing worse for amplitude of 1000nAm compared to 100nAm or 200nAm, because it overestimates the noise when estimating the hyperparameter $\gamma$.

For MxNE and TF-MxNE, the errors shown in Figure~\ref{fig:all_solvers_loc_error} respectively demonstrate an equivalence or a slight improvement when using TF-MxNE compared to MxNE, except for the deepest dipole 8 in red. This is explained by the fact that TF-MxNE is sensitive to the hyperparameters ($\lambda_{space}$, $\lambda_{time}$, window size, and time shift) depending on the SNR and the depth of the dipole. Here we tuned the hyperparameters on a grid search similar for all dipoles, although one might think that the hyperparameters depend on the easiness of the data (so on the SNR, and the depth of each dipole).

The same Figures~\ref{fig:all_solvers_loc_error}-~\ref{fig:all_solvers_ori_error} also show an improvement when using the non-convex version in both irMxNE and irTF-MxNE compared to MxNE and TF-MxNE in localization and orientation, but the biggest difference is mostly seen in amplitude gain (\textit{e.g.} Figure~\ref{fig:all_solvers_amp_error_bst} for Brainstorm-Elekta dataset).

\begin{figure}[th]
	\centering 
    \includegraphics[trim={1cm 1cm 1cm 0cm},width=0.9\textwidth]{benchmark/phantom_errors_compare_all_solvers_ori_error}
    \caption{Comparison of the orientation errors for four different dipoles. \label{fig:all_solvers_ori_error}}
\end{figure}

On the other hand, MNE and dSPM are surprisingly the methods giving the worst results for this dataset. One important argument is the fact that the study is biased as we know that the simulated phantom data is focal/sparse, while MNE and dSPM are not sparse methods. We always take the peak of amplitude and displays the best dipole for each method. sLORETA on the other hand is not a sparse method either, however, it performs much better than MNE and dSPM. The "center" of the pattern estimated with sLORETA is then closer to the exact dipole location compared to the center of dSPM or MNE.

The orientation error is comparable to the location error, where dipole fitting, MUSIC, $\gamma$-map, (ir)MxNE and (ir)TF-MxNE keep being the best performing methods. Dipole fitting is the best when simulating with only one dipole. Note however that dipole fitting does not suffer from the fixed grid resolution of 3.5\,mm used by the other solvers.
Figure~\ref{fig:all_solvers_ori_error} does not show orientation for LCMV, MNE, dSPM, and sLORETA as the orientation is not computed for source estimate. This is due to a current limitation of MNE-Python that only returns the magnitude of the sources when working with free orientation source spaces.

To make the source configurations harder, we considered doing linear combinations of the data produced by different dipoles. However, the locations of the dipoles are nearly the same except that the depth is different. When adding up this type of dipoles, they still have the same pattern in the sensor space, which makes it impossible to disentangle them.

\begin{figure}[p]
	\centering
    \begin{subfigure}{0.9\linewidth}
		\centering
		\includegraphics[trim={1cm 1cm 1cm 1cm},width=\textwidth]{benchmark/phantom_errors_compare_all_solvers_decim_2_loc_error}
	    \caption{\label{fig:all_solvers_loc_error_decim_2}}
    \end{subfigure}
	\hspace{5cm}
	\hfill
    \begin{subfigure}{0.9\linewidth}  
		\centering 
		\includegraphics[trim={1cm 1cm 1cm 1cm},width=\textwidth]{benchmark/phantom_errors_compare_all_solvers_decim_2_ori_error}
		\caption{\label{fig:all_solvers_ori_error_decim_2}}
	\end{subfigure}

		\caption{Comparison of the position and the orientation error between the solvers when taking only one over two epochs to reduce the SNR.\label{all_solvers_loc_error_decim_2}}
\end{figure}

To investigate further the impact of each solver and the impact of a dataset with a high or a low SNR, we show in Figures~\ref{fig:all_solvers_loc_error_decim_2}-~\ref{fig:all_solvers_loc_error_decim_4} the results for the same solvers when we do not take all the epochs but only half of them. Those figures show a deterioration of dipole fitting when using the 20nAm, especially for the case where we keep only one epoch over four (Figure~\ref{fig:all_solvers_loc_error_decim_4}). For 100nAm, 200nAm, and 1000nAm, the difference is small as the data corresponds already to very strong sources. %TF-MxNE still performs better than MxNE due to the kind of simulation done in the phantom, which is basically a sinusoidal waveform easier to be captured by the TF-MxNE.

\begin{figure}[p]
	\centering
    \begin{subfigure}{0.9\linewidth}
		\centering
		\includegraphics[trim={1cm 1cm 1cm 1cm},width=\textwidth]{benchmark/phantom_errors_compare_all_solvers_decim_2_loc_error}
	    \caption{\label{fig:all_solvers_loc_error_decim_4}}
    \end{subfigure}
	\hspace{5cm}
	\hfill
    \begin{subfigure}{0.9\linewidth}  
		\centering 
		\includegraphics[trim={1cm 1cm 1cm 1cm},width=\textwidth]{benchmark/phantom_errors_compare_all_solvers_decim_2_ori_error}
		\caption{\label{fig:all_solvers_ori_error_decim_4}}
	\end{subfigure}

		\caption{Comparison of the position and the orientation error between the solvers when taking only one over four epochs to reduce the SNR.\label{all_solvers_loc_error_decim_4}}
\end{figure}

This analysis has been performed also on two other datasets which lead to the same conclusions (see Figures). The Brainstorm-Elekta dataset had 32 different dipoles, from which we take only four to display in Figure~\ref{phantom_errors_compare_all_solvers_decim_1_bst_loc_error} for both amplitude and orientation errors.

\begin{figure}[p]
	\centering
    \begin{subfigure}{0.9\linewidth}
		\centering
		\includegraphics[trim={1cm 1cm 1cm 1cm},width=\textwidth]{benchmark/phantom_errors_compare_all_solvers_decim_1_bst_loc_error}
	    \caption{\label{fig:phantom_errors_compare_all_solvers_decim_1_bst_loc_error}}
    \end{subfigure}
	\hspace{5cm}
	\hfill
    \begin{subfigure}{0.9\linewidth}  
		\centering 
		\includegraphics[trim={1cm 1cm 1cm 1cm},width=\textwidth]{benchmark/phantom_errors_compare_all_solvers_decim_1_bst_ori_error}
		\caption{\label{fig:phantom_errors_compare_all_solvers_decim_1_bst_ori_error}}
	\end{subfigure}

		\caption{Comparison of the position and the orientation errors between the solvers for Brainstom-Elekta dataset. It shows 4 dipoles among the 32 for a good visibility. Nonetheless, it shows both the dipoles on the surface and the deepest ones.\label{phantom_errors_compare_all_solvers_decim_1_bst_loc_error}}
\end{figure}


Figure~\ref{fig:all_solvers_amp_error_bst} shows the errors in amplitude for Brainstorm-Elekta dataset. The most important point in this figure is to see the difference between the convex MxNE|TF-MxNE and the non-convex irMxNE|irTF-MxNE in terms of amplitude bias. All dipoles basically improve their amplitude estimate when using the non-convex method (irMxNE|irTF-MxNE). For some cases, the irMxNE amplitude estimate is even better than the dipole fitting one and $\gamma$-map. Figure~\ref{fig:all_solvers_amp_error_decim_2_bst} shows the same effect of amplitude bias improvement even when we subsample the epochs, and take only one epoch over two.

The same analysis has been performed on the brainstorm CTF phantom dataset, which contains only one dipole with two different amplitude levels. The take home message remains the same, \textit{i.e.}, dipole fitting is the most suitable solver when no more than one dipole is recorded at a time, however it performs very bad for low SNR. $\gamma$-map has been a very competitive solver compared to (ir)MxNE and (ir)TF-MxNE, but the iterative reweighted solvers remain the best for reducing the amplitude bias, and recovering almost the exact estimated amplitude (around 1\% amplitude error).

This study demonstrates the effectiveness of each method depending on the type of the dipole, \textit{i.e.}, its depth, the level of its amplitude, and the SNR based on the number of epochs. It specifically shows that there is no one best method for all use cases, but rather that the methods are complementary and that some of them works better with some specific conditions. Knowing the type of dataset one has in hand, is then primordial for a good source estimate recovery.

\begin{figure}[p]
	\centering
    \begin{subfigure}{0.9\linewidth}
		\centering
		\includegraphics[trim={1cm 1cm 1cm 1cm},width=\textwidth]{benchmark/phantom_errors_compare_all_solvers_decim_1_bst_amp_error}
	    \caption{\label{fig:all_solvers_amp_error_bst}}
    \end{subfigure}
	\hspace{5cm}
	\hfill
    \begin{subfigure}{0.9\linewidth}  
		\centering 
		\includegraphics[trim={1cm 1cm 1cm 1cm},width=\textwidth]{benchmark/phantom_errors_compare_all_solvers_decim_2_bst_amp_error}
		\caption{\label{fig:all_solvers_amp_error_decim_2_bst}}
	\end{subfigure}

		\caption{Comparison of amplitude error between most of the solvers for 4 different dipoles amond the 32 existing in the Brainstom-Elekta dataset. \label{all_solvers_amp_error_bst}}
\end{figure}

\section{Conclusion \& Perspectives}
In this chapter, the main motivation was to present some results on various source localization techniques applied to phantom data. Being able to investigate "real" datasets with ground-truth is a big privilege to test the large list of existing methods for solving the MEG/EEG inverse problem.

Here we presented some of them, focusing on the approaches defined in this thesis. The conclusion would be that the dipole fitting is the most competitive and best method when having a focal dataset with only one dipole. Unfortunately here, we could not present a phantom dataset with two or more dipoles in the same recording, which would make the source localization more challenging.

A further work would be to investigate this aspect of multiple dipoles. The idea would be to confirm a better performance of convex and non-convex solvers (Variational or Bayesian formulation) compared to dipole fitting or MUSIC.

This chapter also did not show any hyperparameter tuning, or an automatic estimation as shown in Chapter~\ref{chapter:bayesian}, due to the "easiness" of the dataset when having only one dipole simulated. Indeed the location of the strongest dipole as extracted here for evaluation is barely affected by the choice of hyperparameters (provided they stay in reasonable ranges). This would not have been possible if we had two or more simulated dipoles, in which case hyperparameter setting would have been crucial. In this chapter, the hyperparameters to be selected were not having a big impact on the resulted source estimate, except for the TF-MxNE where several ones needed to be fixed. The size of the window size and the time shift of the dictionary were chosen as the best ones from a grid search.

In the near future, we plan to release all the code necessary to replicate these figures (data are already public). And we hope that this will foster new collaborations between researchers working on the MEG/EEG inverse problem. At least it should allow mathematicians and computer scientists working on this problem to more easily compare their methods to the state-of-the-art in the field.
