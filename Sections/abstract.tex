%----------------------------------------------------------------------------------------
%	ABSTRACT PAGE
%----------------------------------------------------------------------------------------
\leavevmode\thispagestyle{empty}\newpage

\begin{abstract}
\addchaptertocentry{\abstractname} % Add the abstract to the table of contents
\Ac{meg} and \ac{eeg} are non-in-\\vasive techniques for investigating human brain activity. They allow the measurement of ongoing brain activation on a millisecond-by-millisecond basis, which makes them very attractive to study the brain dynamics. Since the neuronal activity is measured at a sensor-level distributed over the head, the main question is how can a brain region be identified as the one producing the measured activity with reasonable accuracy? This is the so-called bio-electromagnetic inverse problem which is ill-posed, meaning there is not a unique solution to the problem. The main goal of this thesis is the development of novel methods able to localize in space and time the origin of the observed head surface signals. \\

To do so, very challenging mathematical and computational problems need to be tackled. First of all, since the solution to the ill-posed inverse problem is not unique, constraints need to be set in order to identify an appropriate solution among the multiple possible candidates. The constraints are chosen depending on the assumptions or a priori knowledge based on the characteristics of the source distributions. Common priors are based on the Frobenius norm and lead to a family of methods generally referred to as \ac{MNE}. While these methods have some benefits like simple implementation and robustness to noise, they do not take into account the natural assumption that only a few brain regions are typically active during a specific cognitive task. Interestingly, several source reconstruction techniques have then been proposed, which are based on the assumption to promote focal or \textit{sparse} solutions. These techniques, which are partly used in clinical routine, are suitable, \textit{e.g.} for analyzing evoked responses or epileptic spike activity.\\

%Sparse source reconstruction has been investigated in both standard and time-frequency domains. Source localization in the standard domain implicitly assumes that the active sources are the same ones during the whole time interval of interest, \textit{i.e.} stationary sources. This also implies that if a source is detected as active at one time point, its activation will be non-zero over the entire time window. These solvers in the standard domain promote spatial sparsity only without modeling the dynamics and the non-stationarity of the brain signal. Therefore, they fail to recover realistic source estimates by mixing between time courses of different sources. %the true null activation during baseline (before the onset).
%In the other hand the source localization in the time-frequency domain addresses the problem of having non-stationary focal source activations. Using appropriate sparsity constraints, these solvers promote both spatial and temporal sparsity, which makes them recover better the non-stationary sources and the temporal smoothness. They have been investigated using a tight Gabor dictionary. However, the choice of an optimal dictionary for decomposing the brain signals remains unsolved.  Due to a mixture of signals, \textit{i.e.} short transient signals (right after the stimulus onset) and slower brain waves, the choice of a single dictionary explaining simultaneously both signals' types in a sparse way is difficult.\\

This thesis focuses first on the development of source solvers in the time-frequency domain to promote non-stationary focal source activation. %to overcome the problem of choosing the dictionary.
It introduces a novel method for improving the source estimation relying on a multi-scale dictionary, \textit{i.e.} multiple dictionaries with different scales concatenated to fit short transients and slow waves at the same time. We do not address the problem of learning the dictionary as doing so would make the cost function non-convex, which would deteriorate the speed of convergence, and also make the solver dependent on the initialization. \\%Our novel method was shown to outperform the basic solver in the time-frequency domain in terms of reduced leakage, temporal smoothness and in detection of both signal types.

The second part of this thesis investigates the challenge of estimating hyperparameters involved in the regularization of the inverse problem. In the MEG/EEG community, the compromise between the data fit and the regularization controlled by a hyperparameter is often tuned by hand, which is tedious and time consuming, or it is simply hard coded. This thesis introduces a new way of estimating this hyperparameter automatically when having a synthesis prior.\\
% This work has been based on a former paper of Marcelo Pereyra.

Since source estimates obtained with convex MEG/EEG sparse source imaging are biased in amplitude and often suboptimal in terms of sparsity, iterative reweighted mixed-norm solvers have been proposed in the literature. These solvers make use of non-convex concave penalties in the time or the time-frequency domain. The framework of hierarchical Bayesian modeling (HBM) is a seemingly unrelated approach to encode sparsity. Yet, the next contribution presented in this thesis shows that for certain hierarchical models, a simple alternating scheme to compute fully Bayesian \ac{MAP} estimates leads to the exact same sequence of updates as a standard iterative reweighted strategy (a.k.a. the Adaptive \ac{lasso}). \\

Using simulation and various MEG/EEG datasets, this thesis provides empirical evidence that the novel methods presented here offer promising models and algorithms to improve the estimation of MEG/EEG source activations. A validation of these methods and a comparison with the widely used solvers is also presented using some phantom datasets (\textit{i.e.} actual data recorded with known groundtruth).\\

\textbf{keywords}--- Neuroimaging, magneto/electroencephalography (MEG/EEG), inverse problem, convex/non-convex optimization, sparse regression, multi-scale dictionaries, Gabor transform, hierarchical Bayesian models


\end{abstract}
\leavevmode\thispagestyle{empty}\newpage